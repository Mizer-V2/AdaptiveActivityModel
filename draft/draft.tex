\documentclass{article}
\usepackage{amsmath}

\title{Interactive effects of oxygen and temperature on ectotherms}
\author{Philipp Neubauer \and Ken H. Andersen}

\begin{document}
\SweaveOpts{concordance=TRUE}
\maketitle

\section{Setting up the model}

\subsection{Key assumptions}
\begin{itemize}
\item Animals will adapt acitivty levels to optimise available energy
\item Oxygen is limiting, but not driving energy/mass aquisition
\item Temperature acts dierectly on rates that are determined by enzymatic activity: digestive activity (via maximum consumption) and metabolic cost. The scaling with temperature (determined by the activation energy $E_a$) is assumed the same for these processes.
\item Temperature only acts on ecological rates (food aquisition) via optmisation given temperature driven changes in enzymatic rates.


\end{itemize}

\subsection{Adaptive activity model}

In the adaptive activity model, ectotherms adjust the relative amounts of time ($\tau$) spent foraging and resting to optimise the net energy/mass gain $P_C$. Since both energy gain and loss are sensitive to temperature and oxygen limitations, both the activity level and the net energy gain will be subject to these environmental constraints. Their interplay thus determines available energy for growth, reproduction, and, ultimately, organisms final size.

The model is written in terms of carbon (or energy) and oxygen balance equations:

\begin{align}
f &= \frac{\tau }{\tau  + \frac{h c_T w^{q-p}}{\gamma} } \\
P_C &= (1-\beta-\phi)f h c_T w^q  -(1+\tau \delta)c_T k w^n \\
  &= S_C - D_C \\
P_{0_2} &= f(O_2)w^n - \omega(\beta f h c_T w^q +(1+\tau \delta) c_T k w^n) \\
        &= S_{O_2} - D_{O_2}
\end{align}


where $f$ is the feeding level ([0,1]), determined by the fraction of time spent foraging (or proportion of maximum attack rate) $\tau$, consumption rate $\gamma$ (search rate times prey density) and maximum consumption $h$. In the following, we will refer to $\tau$ as the activity fraction for sake of generality. Maximum consumption, being determined by digestive (enzymatic) processes, is assumed to scale with temperature as $c_T = e^{E_a(T-T_0)/kTT_0}$. Available carbon $P$ is determined by supply ($S_C$) from prey consumption ($f h c_T w^q$), with $\beta$ a loss due to specific dynamic action (SDA, or heat increment; the energy spent absorbing food), and $\phi$ is the exceted fraction of captured food. Metabolic costs ($D_C$) are those of standard metabolism ($k$), as well as active metabolism (scaled in units of standard metabolsim as $\delta k$), with the activity fraction $\tau$ determining the fraction of time that the active metabolism cost applies.

The oxygen budget determines the metabolic scope $S_{0_2}$. Metabolic scope is the difference between oxygen supply $S_{O_2}/w^n=f(O_2)$, the amount of oxgygen supplied per unit weight, and oxygen demand. Demand ($D_{O_2}$) is the sum of oxygen used for SDA ($\beta f h c_T w^q$) and catabolism ($[1+\tau \delta] c_T k w^n$), with $\omega$ determining amount of oxygen required per unit of metabolised carbon. Oxygen supply as a function of ambient oxygen is assumed to follow a saturating function, with $P50$ the point where supply has dropped by 50\% relative to the saturation level $l$. Assuming an asymptotic functional form, we can write $f(O_2)=l(1-e^{O_2\log(0.5)/P50})$.

We now assume that organisms will adjust their activity level to maximise available energy under energetic and oxygen constraints. Energetically, the optimal activity level ($\tau_{opt}$) is found at $\frac{dP}{d\tau}=0$, which gives

\begin{align}
\tau_{opt} = \frac{w^{-n-2 p} \sqrt{(1-\beta-\phi)\gamma^3 \delta k c_T h^2  w^{n+3 p+2 q}}}{\gamma^2 \delta k}-\frac{c_T h w^{q-p}}{\gamma}.
\end{align}

We assume that that the metabolic scope dictates the upper limit of this activity, such that at $\tau_{max}$, oxygen demand $D_{0_2}$ equals total supply $S_{0_2}$. Solving for $\tau_{max}$, the expression is far less elegant:

\begin{math}
     \begin{aligned}
\tau_{max} &= \frac{1}{2 \gamma \delta c_T k \omega} m^{-n-p} \\
           &((-\gamma f(O_2) m^{n+p}+\gamma c_T k \omega m^{n+p}+\delta k c_T^2 h \omega m^{n+q}+\beta c_T \gamma h \omega m^{p+q})^2 - \\ 
           &4 \gamma \delta c_T k \omega m^{n+p} (k h c_T^2 \omega m^{n+q}-f h c_T m^{n+q}))^{0.5} \\
          &\gamma f m^{n+p}-\gamma k \omega c_T m^{n+p}-\delta k h c_T^2 \omega m^{n+q}-\beta \gamma h c_T \omega m^{p+q})
          \end{aligned}
\end{math}

Both temperature and oxygen will influence $\tau$, such that at a given temperature and oxygen concentration, $\tau_{T,O2} = min{\tau_{opt},\tau_{max}}$, meaning we assume that animals will adapt their effort to optimise energy gain ($P_C^{\tau_{T,O2}}$, the net production at $\tau_{T,O2}$).

Assuming a constant investment $r$ in reproduction with weight ($D_{C_r}=rm$), the energy available for growth is directly proportional to $P_C^{\tau_{T,O2}}$, and is thus a function of temperature and oxygen. Furthermore, organism size $m_{\infty}$, determined as the mass $m$ where $S_C^{\tau_{T,O2}} = D_C^{\tau_{T,O2}} + D_{C_r}^{\tau_{T,O2}}$, is also determined by environmental factors.





\end{document}